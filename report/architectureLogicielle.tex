% Intro, divisions en dossiers
\paragraph{}
Le mélange d'affichage et de systèmes multi-agents se prêtait particulièrement
bien à un minimum d'organisation logicielle. Effectivement, s'il est agréable
de faciliter la lecture par une séparation claire de ces deux composantes, il
y a surtout énormément de factorisation possible avec une architecture
adaptée. Tous les objets graphiques nécessitant d'être placés au niveau du sol
par exemple.

\paragraph{}
S'il n'est pas intéressant de détailler tous les choix d'architectures que
nous avons effectué au cours du projet, il existe tout de même quelques points
que nous souhaitons souligner et expliquer.

\subsection{Arborescence des classes d'objets}
% Ajout diagramme de classe réduit?
\paragraph{}
Avec comme principal objectif la factorisation de code, nous avons passé aux
cribles les différentes entités que nous souhaitions créer. Il y avait des
ogres, des robots et des tonneaux. Nous avons tout d'abord établi une
distinction entre les objets qui allaient être manipulés et les agents.
Ceux-ci n'ayant très clairement pas le même contrat à remplir. Cependant, ils
partageaient tout de même clairement une propriété, celle d'être des objets
affichables et mobiles, pouvant être perçu par des agents afin qu'ils prennent
des décisions. C'est cette aspect qui nous a amené à la création de la classe
\verb!GraphicalObject! dont héritaient la classe \verb!Stone! et la classe
\verb!GraphicalAgent!,  la seconde regroupant les ogres et les robots.

\paragraph{}
La classe \verb!GraphicalAgent! encapsulait l'ensemble composé de l'entité et
du nœud MOGRE
correspondant à un objet affichable. Elle définissait des fonctions permettant
de placer les objets sur le sol\footnote{En plaçant le bas de leur
\verb!BoundingBox! sur le plan définis par les axes X et Z dans le repère
global de la scène} ou de les orienter en direction d'un point donné.
Certaines propriétés telles que l'orientation ou l'orientation de la caméra
ont été définies comme virtuelle à cause de la conception des différents
objets, effectivement si l'orientation d'un ogre était colinéaire à son
déplacement, il se déplaçait en fait sur le côté. Il était donc nécessaire
pour certains objets de redéfinir leur orientation afin qu'elle soit
consistante.

\paragraph{}
Un détail que nous avons introduit plus tard était la notion d'utilisabilité,
cette notion est devenu nécessaire à nos yeux lorsque nous avons cherché à
garder des références vers des agents afin de pouvoir les suivre
\footnote{Avec un spot lumineux ou avec une caméra}. D'autres modules
conservant un lien vers cet agent, il était nécessaire de savoir si l'agent
était mort pour cesser de le suivre. De plus cette modification nous a permis
de conserver les tonneaux dans les objets appartenant au monde, sans risquer
pour autant qu'un agent cherche à s'emparer d'un objet qu'un autre agent s'est
déjà approprié.

\subsection{Gestion des entrées}
% Nombre elevé d'actions possibles
\paragraph{}
La possibilité d'agir sur le monde et de contrôler différents facteurs a été
offerte à l'utilisateur, lui permettant ainsi d'activer ou de désactiver le
brouillard par exemple. Le nombre d'options offertes étant relativement
élevé, nous avons pensé qu'il serait très difficile de se retrouver parmi toutes
ces possibilités si nous utilisions une relation simple entre les commandes et leurs raccourcis, aussi bien pour l'utilisateur que pour les développeurs.
Nous avons donc décidé
d'utiliser différents modes, chacun possédant sa propre relation entre les
commandes et les touches utilisées. Nous avons tout de même choisi de partager
certaines commandes entre tous les modes, par exemple le raccourci permettant
de mettre le monde en pause.

\paragraph{}
Afin de pouvoir changer facilement de mode, nous avons utilisé une structure
très simple avec un mode racine auquel on pouvait revenir à tout moment en
pressant la touche \verb!Échap!, les autres modes étaient accessibles depuis
celui-ci via des raccourcis. Afin de pouvoir facilement tester le mode
d'entrée actuel, nous avons aussi créer un enum \verb!InputMode! listant les
différents mode d'entrée.

% CommandHelper + Database
\paragraph{}
Pour que l'utilisateur n'ait pas à aller explorer le code source ou à lire une
notice pour pouvoir comprendre comment accéder aux fonctionnalités désirées,
nous avons écrit une classe \verb!CommandHelper! permettant de décrire
facilement une commande à l'aide de son raccourci et d'une description de son
effet. Pour conserver et pouvoir afficher ces aides facilement et de manière
appropriée, nous avons aussi écrit une classe \verb!CommandDatabase! qui
encapsulait un dictionnaire faisant correspondre à un \verb!InputMode! donné
la liste des commandes associées. Il permettait aussi d'enregistrer celles qui
étaient accessibles depuis tous les modes et offrait grâce à une fonction
l'accès à toutes les commandes\footnote{Globales et spécifiques} disponibles
dans un mode donné.

\subsection{Les overlays}
% Nécessité des overlays + absence héritage
\paragraph{}
Afin de faciliter l'utilisation et le débogage du programme, nous avons assez
rapidement constaté qu'il serait très pratique de pouvoir afficher du texte
dans notre simulation. Il nous est aussi apparu de manière très claire qu'il
n'était pas adapté d'afficher tout le texte dans un seul overlay, les besoins
d'informations n'étant pas les mêmes. Les différents éléments nécessaire
étaient sensiblement identiques, il aurait donc été sage d'utiliser de
l'héritage. Cependant, puisque nous utilisions des scripts pour décrire les
différents overlays et que nous ne savions pas comment utiliser l'héritage à
l'intérieur de ces scripts, nous avons préféré aller au plus simple en
dupliquant du code. Même si notre solution n'était pas du tout élégante d'un
point de vue conception logicielle, elle avait l'avantage d'être
particulièrement rapide à mettre en place\footnote{Au vu des problèmes qu'ont
rencontré d'autres groupes avec l'héritage dans les scripts, notre choix ne
paraît finalement pas si mauvais}. Afin de pouvoir aisément modifier le
placement de tous les overlays simultanément, nous avons tout de même utilisé
un paramètre de Marge commun.

% Débogage
\paragraph{}
Afin de permettre de visualiser facilement les dernières lignes de débogage,
nous avons créé un overlay basé sur l'utilisation d'une {\em fifo}, celui-ci
avait simplement pour but de pouvoir recevoir un nombre quelconque de ligne
sans être saturé.

% Status
\paragraph{}
Certaines questions apparaissant de façon récurrente lors du développement,
il était extrêmement pratique d'avoir un overlay permettant d'afficher
quelques données choisies avec une fréquence de rafraîchissement élevée, sans
pour autant devenir illisible\footnote{Effet que n'aurait pas manqué de
produire un dump de certaines données dans l'overlay de débug}. Nous avons
donc sélectionné les données qui nous semblaient les plus importantes et
nous avons décidé d'un format d'affichage simple qui pouvait être mis à jour
fréquemment sans provoquer de gêne.

% Helper
\paragraph{}
Il est rapidement devenu difficile pour nous de nous souvenir de toutes les
commandes possibles, sachant que la situation serait bien pire pour un
utilisateur que pour un développeur, nous avons créé un layer s'alimentant
sur la \verb!CommandDatabase! pour afficher toutes les commandes possibles
à un moment donné. Cet overlay se met à jour automatiquement lorsque le mode
change, permettant ainsi de limiter le nombre de commande à afficher
simultanément, ce qui améliore sa lisibilité.


\subsection{Utilitaires}
\paragraph{}
Lorsqu'un projet est relativement conséquent, particulièrement s'il permet un
affichage graphique, l'utilisation des sorties standards se révèle bien
souvent insuffisante pour un débogage efficace. Pour ces raisons, nous avons
écrit une classe d'utilitaire permettant de laisser facilement des traces des
différents problèmes.

\paragraph{}
Nous souhaitions donc pouvoir afficher des messages pouvant être utilisés de
deux façons différentes. Soit au cours de l'animation pour mieux comprendre
ce qui est en train de se passer, soit dans des logs pour pouvoir analyser le
comportement à posteriori. Pour cette raison, nous dupliquions les messages à
la fois vers un fichier de log et vers l'overlay de débug.
