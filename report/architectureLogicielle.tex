% Intro, divisions en dossiers
\paragraph{}
Le mélange d'affichage et de systèmes multi-agents se prêtait particulièrement
bien à un minimum d'organisation logicielle. Effectivement, s'il est agréable
de faciliter la lecture par une séparation claire de ces deux composantes, il
y a surtout énormément de factorisation possible avec une architecture
adaptée. Tous les objets graphiques nécessitant d'être placés au niveau du sol
par exemple.

\paragraph{}
S'il n'est pas intéressant de détailler tous les choix d'architectures que
nous avons effectué au cours du projet, il existe tout de même quelques points
que nous souhaitons souligner et expliquer.

\subsection{Arborescence des classes d'objets}
% Ajout diagramme de classe réduit?
\paragraph{}
Avec comme principal objectif la factorisation de code, nous avons passé aux
cribles les différentes entités que nous souhaitions créer. Il y avait des
ogres, des robots et des tonneaux. Nous avons tout d'abord établi une
disctinction entre les objets qui allaient être manipulé et les agents.
Ceux-ci n'ayant très clairement pas le même contrat à remplir. Cependant, ils
partageaient tout de même clairement une propriété, celle d'être des objets
affichables et mobiles, pouvant être perçu par des agents afin qu'ils prennent
des décisions. C'est cette aspect qui nous a amené à la création de la classe
\verb!GraphicalObject! dont héritaient la classe \verb!Stone! et la classe
\verb!GraphicalAgent!,  la seconde regroupant les ogres et les robots.

\paragraph{}
Cette classe encapsulait l'ensemble composé de l'entité et du noeud Mogre
correspondant à un objet affichable. Elle définissait des fonctions permettant
de placer les objets sur le sol\footnote{En plaçant le bas de leur
\verb!BoundingBox! sur le plan définis par les axes X et Z dans le repère
global de la scène} ou de les orienter en direction d'un point donné.
Certaines propriétés telles que l'orientation ou l'orientation de la caméra
ont été définies comme virtuelle à cause de la conception des différents
objets, effectivement si l'orientation d'un ogre était colinéaire à son
déplacement, il se déplaçait en fait sur le côté. Il était donc nécessaire
pour certains objets de redéfinir leur orientation afin qu'elle soit
consistante.

\paragraph{}
Un détail que nous avons introduit plus tard était la notion d'utilisabilité,
cette notion est devenu nécessaire à nos yeux lorsque nous avons cherché à
garder des références vers des agents afin de pouvoir les suivre
\footnote{Avec un spot lumineux ou avec une caméra}. D'autres modules
conservant un lien vers cet agent, il était nécessaire de savoir si l'agent
était mort pour cesser de le suivre. De plus cette modification nous a permis
de conserver les tonneaux dans les objets appartenant au monde, sans risquer
pour autant qu'un agent cherche à s'emparer d'un objet qu'un autre agent s'est
déjà approprié.

\subsection{Gestion des entrées/sorties}

\subsection{Les overlays}

\subsection{Utilitaires}
