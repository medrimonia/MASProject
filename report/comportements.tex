\paragraph{}
Afin d'explorer un peu l'application de modèles comportementaux sur des
systèmes multi-agents, nous avons décidé d'additionner plusieurs modèles
de comportements simples. Sans utiliser d'entité contrôlant l'ensemble des
agents, le but est d'obtenir des effets de groupe notables. L'idée principale
étant de donner l'illusion que les agents participent à une tâche commune
avec une vision globale alors qu'ils font leurs choix seuls, avec uniquement
une connaissance locale.

\paragraph{}
Afin d'éviter que les agents effectuent leurs décisions et leurs actions dans
le même ordre à cause de l'aspect séquentiel de l'exécution, nous avons décidé
que nous allions simplement prendre une copie de la liste mélangée de façon
uniforme.

\subsection{Déplacement d'objets}
%Modèle probabiliste
%Dualité ogre/robots
\subsection{Sexué}
%Genre
%Gestion de priorité
%Densité critique
% -> Effets de localisation
\subsection{Communication}
%Échange d'information
%Schrei nach liebe
