\paragraph{}
Afin d'explorer un peu l'application de modèles comportementaux sur des
systèmes multi-agents, nous avons décidé d'additionner plusieurs modèles
de comportements simples. Sans utiliser d'entité contrôlant l'ensemble des
agents, le but est d'obtenir des effets de groupe notables. L'idée principale
étant de donner l'illusion que les agents participent à une tâche commune
avec une vision globale alors qu'ils font leurs choix seuls, avec uniquement
une connaissance locale.

\paragraph{}
Afin d'éviter que les agents effectuent leurs décisions et leurs actions dans
le même ordre à cause de l'aspect séquentiel de l'exécution, nous avons décidé
que nous allions simplement prendre une copie de la liste mélangée de façon
uniforme.

\subsection{Déplacement d'objets}
%Modèle fourmi/brindille
\paragraph{}
Nous avons décidé de commencer par implémenter un des modèles de système
multi-agent classique qui nous avait été présenté en cours. Dans celui-ci, les
agents se contentent de se déplacer aléatoirement, jusqu'à ce qu'ils
rencontrent une ressource qu'ils ramasseront pour la déposer dès qu'ils se
retrouvent face à une autre. Le comportement complexe qui doit émerger de ces
action simple est le fait que les ressources vont converger vers un seul tas.
Nous avons choisi de représenter les fourmis par des ogres et les ressources par
des tonneaux.

% Ramassage/Dépose à l'infini
\paragraph{}
Lorsque nous avons tenter d'implémenter ce modèle, nous nous sommes retrouvés
face au problème suivant~: Les ogres ramassaient les tonneaux et les
relâchaient tout de suite après dès qu'il y avait plusieurs tonneaux autour
d'eux. Afin de remédier à ce problème, nous avons effectué une recherche sur
internet et nous avons trouvé une solution raffinée et décrite en détail
\footnote{\url{http://liris.cnrs.fr/simon.gay/index.php?page=sma&lang=en}}.
Celle-ci étant prévue initialement pour plusieurs ressources différentes. Nous
avons conservé le principe général qui est d'avoir une probabilité de ramasser
le tonneau inversement proportionnelle au nombre de tonneau dans le voisinage. À
l'opposé, la probabilité de déposer un tonneau est proportionnelle au nombre
de tonneau dans le voisinage. Une fois cette solution implémentée, la rapidité
de la convergence s'est grandement améliorée.

%Dualité ogre/robots
\paragraph{}
Afin de rendre la tâche des ogres plus difficiles, nous leur avons créé des
adversaires, les robots. Notre idée était relativement simple, en copiant le
comportement des ogres, mais en inversant les probabilités, il était possible
d'obtenir des agents qui cherchent à éparpiller les tonneaux. Nous avons donc
pu observer que si le nombre d'ogre était trop faible face au nombre de robots,
il n'y avait pas de convergence des ressources vers des tas.


\subsection{Sexué}
%Genre
\paragraph{}
Afin qu'il y ait une certaine hétérogénéité parmi les agents, nous avons décidé
d'attribuer aléatoirement un genre à la création des ogres. Cet aspect
permettait d'ouvrir la voie à un comportement reproductif. Ce comportement était
aussi d'une grande simplicité. Lorsque les femelles sentent le besoin de se
reproduire, elles cessent leurs déplacements et attendent qu'un mâle vienne les
inséminer.

\paragraph{}
Un des principaux intérêts de l'ajout de ce comportement était pour nous le fait
qu'il rentrait en conflit avec le comportement initial. Il nous a donc été
nécessaire de prendre en compte cet aspect, afin qu'une décision soit prise par
les agents. Nous avons tranché en décidant que le comportement sexuel serait
toujours prioritaire sur le déplacement de tonneau. Il aurait aussi été
intéressant d'essayer de faire varier les intérêts suivant les agents, simulant
ainsi une forme de caractère. Cet aspect aurait aussi présenté l'intérêt
d'éviter que tous les ogres rentrent dans un comportement identique par exemple,
diminuant ainsi la probabilité qu'une tâche soit excessivement privilégiée.

\paragraph{}
Le problème de ce comportement sexué était qu'il engendrait une situation
hautement instable. Effectivement, plus les ogres étaient nombreux, plus le
potentiel de reproduction était élevé et plus la population d'ogre croissait.

\paragraph{}
Afin d'obtenir des situations stables, nous avons décidé que chaque ogre
évaluerait en permanence la densité de son environnement. Pour ce faire, il
effectuait simplement une moyenne mobile sur la taille de son voisinage. Si
cette densité était supérieure à un certain seuil, l'ogre ne souhaitait pas se
reproduire, estimant que son espèce était déjà trop nombreuse et qu'il valait
mieux ne pas empirer les choses.

\paragraph{}
Outre la limitation de la population, ce choix des agents avait un autre impact,
il favorisait la reproduction dans les zones les moins peuplées. Au vu de la
taille de notre monde par rapport à la vitesse des ogres, cet aspect n'était pas
flagrant. Une simulation dans un monde beaucoup plus grand avec d'autres lois de
déplacement aurait sûrement permis d'illustrer cette propriété avec beaucoup
plus clairement.

\subsection{Communication}
%Échange d'information
%Schrei nach liebe
