\paragraph{}
Ce projet a pour but de développer au travers d'une simulation 3D, un
ensemble de comportement réalisé par un groupe d'individu. Dans notre cas il
s'agira de simuler au travers de tête d'ogre , et de robot, l'accumulation de
nourriture dans une colonie de fourmis. Pour cela nous utiliserons le
\verb!C#! et plus précisément le wrapper MOGRE pour créer une simulation
graphique permettant d'observer concrètement ces comportements.

\paragraph{}

Notre projet sera donc divisé en deux parties, la première concernera 
le comportement des différents agents, la deuxième sera axé sur la simulation
graphique et les différents overlays intégré à l'application. Enfin une partie
supplémentaire traitera de l'illusion de vie crée pour les ogres.

\paragraph{}
Plutôt que de partir d'un projet vide et d'y incorporer peu à peu des
fonctionnalités, nous avons décidé de commencer avec le projet fourni pour
les tutoriels de MOGRE et de remplacer peu à peu les fonctionnalités qu'il
incluait. En procédant de la sorte, nous nous assurions d'avoir toujours un
résultat facilement visualisable et nous pouvions de plus revenir facilement aux
tutoriels afin de nous en inspirer pour créer quelque chose qui convenait plus à
nos envies.

\paragraph{}
Finalement, il est important de noter pour la suite du rapport, que les
ressources que nous avons utilisées étant initialement des pierres et s'étant
par la suite transformée en tonneaux, ces deux termes seront ici considéré comme
équivalents.
