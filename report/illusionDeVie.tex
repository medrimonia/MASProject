%Temporalité de la vie
\paragraph{}
Nous avons évoqué précédemment notre envie de contribuer à l'illusion de
réalité en incluant un cycle de jour et de nuits dans la luminosité globale.
En poursuivant le même objectif, nous avons pensé qu'il était intéressant de
simuler une évolution des agents dans le temps, ce qui permet de leur
donner un aspect plus vivant. Par ailleurs, considérer ce facteur offrait
aussi d'autres possibilités pour le comportement, entraînant des modifications
des phénomènes émergents de la simulation.

\subsection{Croissance}
\paragraph{}
Nous avons choisi d'illustrer cette temporalité de la vie par un aspect visuel
facile à percevoir. Nous avons pensé que l'idéal était de simuler la
croissance des ogres jusqu'à ce qu'ils atteignent l'âge adulte. Nous avons
donc établi une taille à la naissance, une taille adulte et un temps de
croissance. Afin de simplifier les calculs, nous avons établi que la taille
entre ces deux âges était déterminée par une simple interpolation linéaire.

\paragraph{}
Cette légère modification a tout de même entraîné la nécessité de replacer les
ogres sur le sol après chaque modification de taille, car lorsque le facteur
de taille changeait, il était nécessaire de mettre à jour la position de
l'ogre selon l'axe Y.

\subsection{Maturité sexuelle}
\paragraph{}
Afin d'apporter un brin supplémentaire de réalisme et d'éviter une
augmentation de la population trop rapide, nous avons décidé de définir un âge
minimal pour la fertilité des mâles et des femelles. Sans cette limite,
lorsqu'une femelle accouchait dans un lieu isolé, il était envisageable
qu'elle souhaite rapidement se reproduire et que le mâle le plus proche soit
son fils. Notre ajout permettait donc aussi d'éviter ce genre de situation,
même si au vu de la faible population d'ogre que nous avions, la consanguinité
resterait très difficile à éviter.

\paragraph{}
Puisque nous avions installé un âge minimum pour procréer, il nous a paru
naturel d'inclure aussi une limite supérieure. Outre les aspects déjà
mentionnés, ce paramètre permettait d'assurer que même si le nombre d'ogre
devenait faible, ceux qui n'étaient pas aptes à procréer continuait à
travailler. En revanche, cet ajout n'avait pas un effet positif en matière de
stabilité de la population. Effectivement, il était possible d'avoir une
densité d'ogre élevée sans qu'aucun n'ait la capacité de se reproduire, même
si avec nos paramètres finaux ce cas n'apparaissait pas de manière flagrante.

\subsection{Mort}
