%Temporalité de la vie
\paragraph{}
Nous avons évoqué précédemment notre envie de contribuer à l'illusion de
réalité en incluant un cycle de jour et de nuits dans la luminosité globale.
En poursuivant le même objectif, nous avons pensé qu'il était intéressant de
simuler une évolution des agents dans le temps, ce qui permet de leur
donner un aspect plus vivant. Par ailleurs, considérer ce facteur offrait
aussi d'autres possibilités pour le comportement, entraînant des modifications
des phénomènes émergents de la simulation.

\subsection{Croissance}
\paragraph{}
Nous avons choisi d'illustrer cette temporalité de la vie par un aspect visuel
facile à percevoir. Nous avons pensé que l'idéal était de simuler la
croissance des ogres jusqu'à ce qu'ils atteignent l'âge adulte. Nous avons
donc établi une taille à la naissance, une taille adulte et un temps de
croissance. Afin de simplifier les calculs, nous avons établi que la taille
entre ces deux âges était déterminée par une simple interpolation linéaire.

\paragraph{}
Cette légère modification a tout de même entraîné la nécessité de replacer les
ogres sur le sol après chaque modification de taille, car lorsque le facteur
de taille changeait, il était nécessaire de mettre à jour la position de
l'ogre selon l'axe Y.

\subsection{Maturité sexuelle}
\paragraph{}
Afin d'apporter un brin supplémentaire de réalisme et d'éviter une
augmentation de la population trop rapide, nous avons décidé de définir un âge
minimal pour la fertilité des mâles et des femelles. Sans cette limite,
lorsqu'une femelle accouchait dans un lieu isolé, il était envisageable
qu'elle souhaite rapidement se reproduire et que le mâle le plus proche soit
son fils. Notre ajout permettait donc aussi d'éviter ce genre de situation,
même si au vu de la faible population d'ogre que nous avions, la consanguinité
resterait très difficile à éviter.

\paragraph{}
Puisque nous avions installé un âge minimum pour procréer, il nous a paru
naturel d'inclure aussi une limite supérieure. Outre les aspects déjà
mentionnés, ce paramètre permettait d'assurer que même si le nombre d'ogre
devenait faible, ceux qui n'étaient pas aptes à procréer continuait à
travailler. En revanche, cet ajout n'avait pas un effet positif en matière de
stabilité de la population. Effectivement, il était possible d'avoir une
densité d'ogre élevée sans qu'aucun n'ait la capacité de se reproduire, même
si avec nos paramètres finaux ce cas n'apparaissait pas de manière flagrante.

\subsection{Mort}
% Pourquoi faut-il tuer les ogres?
\paragraph{}
Étant donné que nous avions inclus la procréation que nous inhibions celle-ci
en fonction de la densité d'ogre perçue par un agent, nous serions rapidement
arrivé à une certaine stabilité avec uniquement des ogres adultes si nous
n'avions pas inclus la possibilité que ceux-ci meurent.

% Non déterminisme
\paragraph{}
Pour éviter trop de déterminisme, nous avons décidé que nous ne nous
contenterions pas d'une simple date de mort qui serait fixe et communes pour
tous les ogres. Nous aurions aussi pu décider de déterminer la durée de vie
de ceux-ci à leur création, mais nous avons préférer déterminer dynamiquement
à chaque tour si les ogres mourraient.

\paragraph{}
Malgré cet aspect probabiliste, nous souhaitions rester réaliste en conservant
une sorte d'âge limite et de plus, nous voulions que les jeunes ogres aient
moins de risque de mourir. Afin de ne pas être influencée par le fps, la
probabilité de mourir devait donc dépendre aussi du temps qui s'était écoulé.
Nous avons donc fait une petite étude mathématique pour déterminer la fonction
à utiliser.

\paragraph{}
Si l'on défini $p(t)$ comme étant la probabilité qu'un ogre meure exactement
à l'instant $t$ et $t_{max}$ comme la durée de vie maximal d'un ogre, on
peut écrire facilement~:
$$\int_0^{t_{max}} p(t) \mathrm dt = 1$$
Nous avons décidé d'utiliser une fonction $p(t)$ de la forme $a \cdot t^3$
afin que la probabilité de mourir avant d'atteindre la maturité sexuelle soit
très faible. On obtient donc comme primitive $P(t) = \frac{a \cdot t^4}{4}$,
La valeur de $a$ peut être déterminée par la première équation~:
$$P(t_{max}) - P(0) = 1$$
$$\frac{a \cdot {t_{max}}^4}{4} = 1$$
$$ a = \frac{4}{{t_{max}}^4}$$
En remplaçant $a$, on obtient donc~:
$P(t) = \frac{t^4}{{t_{max}}^4} = \left ( \frac{t}{t_{max}} \right ) ^4$, ce qui
permet d'exprimer la probabilité qu'un ogre meure dans l'intervalle de temps
de $t_a$ à $t_b$~:
$$\int_{t_a}^{t_b} p(t) \mathrm dt = P(t_b) - P(t_a)$$
$$\int_{t_a}^{t_b} p(t) \mathrm dt = \left ( \frac{t_b}{t_{max}} \right ) ^4 - \left ( \frac{t_a}{t_{max}} \right ) ^4$$

\paragraph{}
Nous avons donc pu utiliser ce calcul dans notre code afin de déterminer si
l'ogre mourait dans un intervalle de temps donné. Cette fonction permettait
donc de répondre correctement à toutes les envies que nous avions exprimées
tout en prenant un temps de calcul négligeable.
