% Richesse de la réalité virtuelle + sma
\paragraph{}
La liberté de développement qui nous a été octroyée pour ce projet nous a
permis de nous rendre compte à quel point le monde de la réalité virtuelle et
celui des systèmes multi-agents sont riches. Effectivement, pour chaque
fonctionnalité que nous terminions, trois nouvelles idées d'améliorations
germaient. Il était donc difficile de conserver une liste des ajouts que nous
souhaitions faire et il semble clair que même si nous avions eu beaucoup plus
de temps à consacrer au projet, nous l'aurions tout de même terminé en
songeant à toutes les fonctionnalités qui auraient pu être ajoutées.

\paragraph{}
Parmi les possibilités d'améliorations, on peut songer notamment à une
amélioration de l'aspect du ciel, à une fiabilisation de la caméra en mode
poursuite ou encore à une apparition et disparition progressive du brouillard
sans intervention de la part de l'utilisateur. Mais il est aussi possibles de 
penser à des améliorations concernant le système multi-agents. L'ajout de
ressources nécessitant que les agents collaborent pourrait être
particulièrement intéressant, de même qu'un agrandissement de l'espace et du
nombre d'agents couplées à l'ajout de murs, ce qui permettrait probablement
d'illustrer les aspects de reproduction localisée évoqués précédemment.

\paragraph{}
C'est précisément en tentant de changer les dimensions du monde ainsi que le
nombre d'agents que nous avons pu observer une chute drastique des
performances. Au vu de cet effet attendu, nous avons pensé que, s'il est
intéressant de mélanger la réalité virtuelle et les systèmes multi-agents, se
focaliser sur une seule de ces problématiques présente aussi des avantages,
tels que l'utilisation d'un plus grand nombre d'agents ainsi que la
possibilité d'effectuer du calcul intensif afin d'observer le temps de
convergence en fonction de certains paramètres par exemple.
