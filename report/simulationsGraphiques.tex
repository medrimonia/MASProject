\subsection{Animation des robots}

%Pourquoi une telle animation, cohérence ou simple amusement?
\paragraph{}
Il nous a semblé important que pour des problèmes de cohérence dans la
simulation , la tête d'ogre qui ne possédait pas de jambes n'avait pas besoin
d'animation particulière mais que les robots qui eux en possédaient se devaient de
ne pas se déplacer uniquement par glissement sur le sol. Pour cela nous avons
décidé d'utiliser l'animation fournit avec le mesh du robot. 

Pour cela nous avions besoin de définir:\\
\begin{itemize}
  \item Une direction
  \item Une destination
  \item Une distance à l'objet
\end{itemize}

% Utilité de ces variables
\paragraph{}
Ces grandeurs nous permettaient de déterminer l'objectif du robot. Nous pouvions
ensuite nous en servir pour déterminer la position du robot sur le sol grâce à
la méthode \verb!Translate! de la classe \verb!Node!.

\subsection{Caméras}
\subsection{Lumières}


\subsection{Brouillard}
\subsection{Ciel}
