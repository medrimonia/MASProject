\subsection{Animation des robots}

%Pourquoi une telle animation, cohérence ou simple amusement?
\paragraph{}
Il nous a semblé important que pour des problèmes de cohérence dans la
simulation , la tête d'ogre qui ne possédait pas de jambes n'avait pas besoin
d'animation particulière mais que les robots qui eux en possédaient se devaient de
ne pas se déplacer uniquement par glissement sur le sol. Pour cela nous avons
décidé d'utiliser l'animation fournit avec le mesh du robot. 

Pour cela nous avions besoin de définir:\\
\begin{itemize}
  \item Une direction
  \item Une destination
  \item Une distance à l'objet
\end{itemize}

% Utilité de ces variables
\paragraph{}
Ces grandeurs nous permettaient de déterminer l'objectif du robot. Nous pouvions
ensuite nous en servir pour déterminer la position du robot sur le sol grâce à
la méthode \verb!Translate! de la classe \verb!Node!.

\subsection{Caméras}
\subsection{Lumières}

Afin de rendre plus réaliste le modèle de simulation au niveau de
l'environnement, il a été choisi de développer des modèles environnementaux en
supplément des modèles de comportement des entités vivantes. Pour cela nous
avons réalisé un système de modification de la luminosité en fonction du temps
passé sur la simulation.

\paragraph{}

Ce système utilise l'implémentation de la notion de temporalité dans le projet.
Cette notion va nous permettre de calculer l'heure dans le modèle et ainsi
pouvoir déterminer en fonction de l'heure et du temps de chaque phase, la
d'intensité lumineuse dans lequel on se trouve.

\paragraph{}

% Manque un peu de précision sur ce qu'on fait , explication algo?

\paragraph{}

Deux autres modes ont été implémenté:\\
\begin{itemize}
  \item Un mode de nuit exclusive
  \item Un mode de jour exclusif\\
\end{itemize}

Ces modes sont accessibles grâce à l'overlay présenté dans la partie %ref
%overlay. 

\subsection{Brouillard}

\paragraph{}
Comme nous l'avons vu précédemment, il a été décidé de réalisé des évènements
sur l'environnement de la simulation. La temporalité utilisé à travers
l'alternance jour/nuit nous a permis de simuler le déroulement normal d'une
journée. Cependant il ne fait pas toujours soleil, pour palier aux aléas de la
météo, il a été choisi d'intégrer un brouillard à la simulation.

\paragraph{}
Il est possible grâce à diverses commandes , de modifier les différents
paramètres du brouillard (intensité, type, présence\ldots), ces différentes
commandes sont toutes listées dans l'overlay 


\subsection{Ciel}
