\subsection{Animation des robots}

%Pourquoi une telle animation, cohérence ou simple amusement?
\paragraph{}
Il nous a semblé important que pour des problèmes de cohérence dans la
simulation, la tête d'ogre qui ne possédait pas de jambes n'avait pas besoin
d'animation particulière mais que les robots qui eux en possédaient se devaient de
ne pas se déplacer uniquement par glissement sur le sol. Pour cela nous avons
décidé d'utiliser l'animation fournie avec le mesh du robot. 

Pour cela nous avions besoin de définir:\\
\begin{itemize}
  \item Une direction
  \item Une destination
  \item Une distance à l'objet
\end{itemize}

% Utilité de ces variables
\paragraph{}
Ces grandeurs nous permettaient de déterminer l'objectif du robot. Nous pouvions
ensuite nous en servir pour déterminer la position du robot sur le sol grâce à
la méthode \verb!Translate! de la classe \verb!Node!.

\subsection{Caméras}
%Remplacement de la caméra
\paragraph{}
Bien que le déplacement de la caméra soit disponible grâce au code de tutoriel
de Mogre, nous avons décidé de nous en passer et de réimplémenter la notre.
Outre le fait que cela présente un intérêt pédagogique évident, le gestionnaire
d'entrées du tutoriel utilisait des touches que nous souhaitions utiliser pour
d'autres fonctionnalités. Nous souhaitions aller plus loin et apporter plus
d'effets, mais nous avons commencé par permettre uniquement une navigation
comportant des déplacement et des rotations de la caméra actuelle.

\paragraph{}
Nous avons apporté des modifications afin de permettre l'utilisation de caméra
en mode poursuite sur des ogres. Pour cela, nous avons du redéfinir la méthode
donnant l'orientation d'une caméra pour les ogres, la méthode générique étant
inadaptée. Nous reculions ensuite la caméra suffisamment pour que l'affichage ne
soit pas perturbé par l'ogre.

\paragraph{}
Nos ogres changeant de direction soudainement et les changements de caméra étant
naturellement assez brutaux, nous avons cherché à les lisser, en faisant de
l'interpolation entre la position initiale et la cible. Après plusieurs essais
infructueux avec les quaternions et les matrices de rotation, nous avons réussi
à obtenir un résultat satisfaisant \footnote{Même s'il n'est pas dépourvu de
bugs} en nous servant d'une méthode expérimentale à base de moyennes mobiles sur
les axes décrivant le repère source et le repère de destination.

\paragraph{}
Cette utilisation de moyennes mobiles permet aussi de lisser les changements de
position et d'orientation de caméra lorsque l'ogre suivi change de direction,
même si elle ne suffit lorsqu'ils sont vraiment trop violents et répétés.

\subsection{Lumières}
\paragraph{}
Afin de rendre plus réaliste le modèle de simulation au niveau de
l'environnement, il a été choisi de développer des modèles environnementaux en
supplément des modèles de comportement des entités vivantes. Pour cela nous
avons réalisé un système de modification de la luminosité en fonction du temps
passé sur la simulation.

\paragraph{}
Ce système utilise l'implémentation de la notion de temporalité dans le projet.
Cette notion va nous permettre de calculer l'heure dans le modèle et ainsi
pouvoir déterminer en fonction de l'heure et du temps de chaque phase,
l'intensité de la lumière ambiante.

\paragraph{}
Une autre fonctionnalité implémentée liée à la lumière est le tracking d'ogre.
Ce dernier intervient lors de la recherche d'un ogre en particulier, il permet
d'allumer un spot puissant qui éclairera un ogre en le poursuivant au fur et à
mesure de ses déplacements.

\paragraph{}
Deux autres modes ont été implémenté:\\
\begin{itemize}
  \item Un mode de nuit exclusive
  \item Un mode de jour exclusif\\
\end{itemize}

\paragraph{}
Les différentes commandes ainsi que leurs raccourcis sont listés dans l'overlay
d'aide lorsque ce mode est activé.

\subsection{Brouillard}

\paragraph{}
Comme nous l'avons vu précédemment, il a été décidé de réalisé des évènements
sur l'environnement de la simulation. La temporalité utilisé à travers
l'alternance jour/nuit nous a permis de simuler le déroulement normal d'une
journée. Cependant il ne fait pas toujours soleil, pour palier aux aléas de la
météo, il a été choisi d'intégrer un brouillard à la simulation.

\paragraph{}
Il est possible grâce à diverses commandes, de modifier les différents
paramètres\footnote{Intensité, type de brouillard et activation} du brouillard.

\paragraph{}
Les différentes commandes ainsi que leurs raccourcis sont listés dans l'overlay
d'aide lorsque ce mode est activé.


\subsection{Terre et Ciel}

\paragraph{}
La dernière partie de la réalisation d'un environnement plus proche du réel fut
la réalisation d'un sol et d'un ciel autre qu'une dalle blanche et d'un ciel
noir.

\paragraph{}
Pour cela nous avons décidé d'assigner un ciel de type Dôme qui permet lors du
suivi vidéo à la troisième personne, d'avoir l'impression d'un ciel réel.
Nous avons aussi activé un terrain simulant un sol pavé de pierre grise.
